\documentclass{ctexbeamer}

\usetheme{Madrid}      % 主题
\usecolortheme{dolphin}

% 页码显示
\setbeamertemplate{footline}[frame number]

% 移除右下角默认导航符号
\setbeamertemplate{navigation symbols}{}

% 上方点点导航(只显示章节)
\setbeamertemplate{headline}{%
  \begin{beamercolorbox}[wd=\paperwidth,ht=2.5ex,dp=1.125ex]{section in head/foot}
    \vskip1pt
    \insertsectionnavigationhorizontal{\paperwidth}{}{}{}
  \end{beamercolorbox}
}

\title{凸函数初探}
\author{谭富炫}
\date{\today}

\begin{document}

\frame{\titlepage}

\section{凸函数与凸包}
\begin{frame}{什么事凸性}

回忆在数学中,我们称一个函数是凸的,如果其二阶导 f''(x)>0,或者说一阶导递增。

\includegraphics[width = .3\textwidth]{D://Polynomialdeg2.jpg}

如图是一个典型的二次函数

注意函数的凸性描述的是「上境图」的性质(即函数上方点的集合)

约定:凸函数=\textbf{下凸}函数,凹函数=\textbf{上凸}函数。这里国内外的定义似乎相反,以通用定义为标准。
\end{frame}
\begin{frame}{凸函数的性质}
对于任意两个凸函数 f(x),g(x),max(f(x),g(x)) 也是凸函数。

同理,对于任意两个凹函数,其 min 也是凹函数。

特别的,直线既是凸函数也是凹函数

\end{frame}
\begin{frame}{离散情况}
在 OI 中,更常见的是离散情况。

如果从数列的角度出发:我们称一个数列是凸的,如果其二阶差分 $\ge 0$。

如果从函数的角度出发:则凸函数由若干条斜率递增的直线拼起来,形如分段函数。尽管在连接点处函数不可导,但我们认为其二阶导即为前后两段斜率之差。

离散情况满足连续情况的大部分性质
\end{frame}

\begin{frame}{什么事凸包}
\includegraphics[width = .4\textwidth]{D://ConvexHull.png}

凸包是一个点集 $\mathcal C=\{(x,y)\}$

满足任意两点的连线都在凸包内 $\forall p_1,p_2\in \mathcal C,t\in(0,1),p_1+(p_2-p_1)t\in\mathcal C$

找到最左边(x 最小)的点和最右边(x 最大)的点,可以把凸包分为上下两个凸包。

那么上凸包的边,从左往右斜率递减;下凸包的边,从左往右斜率递增。

如果我们连接定义域开头和结尾两个点,那么可以把凸函数也看成凸包。
\end{frame}
\begin{frame}{维护凸包的算法}
求凸包最常用的算法是 Graham 扫描法

OI-wiki 上有动图(我小时候都没有)

凸包最有用的一点是可以做 Mincowski 合并。

可以证明对于两个凸包 $\mathcal{C}_1,\mathcal{C}_2$,其 Mincowski 和 $\{p_1+p_2|p_1\in \mathcal{C}_1,p_2\in\mathcal{C}_2\}$ 也是一个凸包。

求 Mincowski 和最常用的算法是归并排序,可以证明两个凸包的 Mincowski 和的边恰为原两个凸包的边按顺序拼起来的。

如果边提前有序,那么时间复杂度是 $\mathcal O(n)$ 的,否则需要 $\mathcal O(n\log n)$ 排序。

OI-wiki 上有讲(我小时候都没有)。

板子:[JSOI2018]战争

凸包还可以动态维护,参见 CF70D,不过要写平衡树,一般不会考这个。

二分凸包可以求出凸包与直线的切点。

半平面交得到的也是凸包。

\end{frame}

\begin{frame}{维护凸函数的算法}
李超树是一种经典的维护凸函数的算法。

OI-wiki 貌似没有提到它的凸性,其实若干条线段取 max/min 本身就是凸的了。

可以把凸函数转化为凸包进行维护,凸函数的 max-plus 卷积即为凸包的 Mincowski 和。

(max-plus 卷积指 $h(x)=\max_{0\le y\le x}\{f(y)+g(x-y)\}$)

\end{frame}
\begin{frame}

还有一种小清新的维护凸函数的数据结构——slope trick。

考虑维护一个 set 或者其它的东西,包含 $\Delta^2 a_i$ 个 i。

这里 $\Delta^2$ 表示二阶差分,$\Delta^2 a_i=\Delta a_{i+1}-\Delta a_i= a_{i+2}-2a_{i+1}-a_i$。

或者说每出现斜率变化就把这个变化的幅度插入进去。

不妨假设斜率从 0 开始

那么可以发现:

\begin{itemize}
  \item 求 $\Delta a_i$,即为 $0\sim i-1$ 的元素的数量
  \item 两个凸函数的 max-plus 卷积,即为它们的 slope trick 的可重并。
  \item 特别的,如果你想要跟一个线段做 max-plus 卷积,即插入一条边,相当于 set 插入元素。
\end{itemize}

\end{frame}

\begin{frame}{凸性证明}

除了用定义证明或者瞪眼,还可以用网络流的方式证明。

如果你发现答案可以转写成一个费用流的形式,那么它肯定关于流量是一个凹函数。

同时,打表也不失为一种好方法。

\end{frame}

\begin{frame}{更高维度}

前面提到半平面交得到的一定是凸包,即满足所有 $a_ix_i+b_iy_i\le c_i$ 的点集。

如果推广到更高维度,在 $n$ 维空间中,存在若干半超空间的限制 $\sum_{j=1}^{n}a_{i,j}x_j\le b_i$,求其关于一个线性函数 $f(x)=\sum_{i=1}^n v_ix_i$ 的最值,即为所谓「线性规划」的问题。

算法竞赛中,线性规划的 practical 的算法是单纯形法(simplex),时间复杂度是 $\mathcal O$(玄学),最坏是阶乘级别的,可以用来骗分。

\end{frame}

\begin{frame}{WQS 二分}
  
  设想你有一个隐藏的凸函数 $f(x)$,你希望求出它的某个点的值 $f(x_0)$,但是你只有求出其全局最小值 $\min f(x)$ 的方法。
  
  但是你还发现你还能求出 $\min \{f(x)-kx\}$,所以可以用二分凸包的方式求出 $f(x_0)$,只是此时我们可以改变的是 $k$ 而非切点 $x_0$。
  
  这就是所谓的 WQS 二分

\end{frame}

\section{例题}

\begin{frame}{QOJ9119}

  给你一张连通无向图,有重边无自环,请你判断是否存在一种给边赋权的方式 $w_i\in \{0,1,\dots,L\}$,使其\textbf{任意}生成树的权重恰为 $W$。

  如果存在这样的权重分配,求所有满足条件的分配中平方和 $\sum w_i^2$ 的最小值。

  给出 $K,L$,对于所有 $W=0,1,\dots,K$ 求出答案。

  $N,L,K\le 10^5,M\le 2\times 10^5$

\end{frame}

\begin{frame}
  考虑建出原图的一棵生成树,然后我们发现对于任意一条非树边,应当要求其树上路径上任意一条边的权值都与其相等。

  我们不断地缩下去,最终发现对于每个点双内部的边的权值都应当相同。

  所以建出圆方树。记总共有 $C$ 个点双,第 $i$ 个点双有 $a_i$ 条树边,$b_i$ 条边,那么限制形如:
  \[
  \begin{aligned}
  \mathrm{minimize}\quad & \sum_{i=1}^C b_i x_i^2 \\
  \text{s.t.}\quad & \sum_{i=1}^C a_i x_i = W, \\
                    & 0 \le x_i \le L, \forall i
  \end{aligned}
  \]
  然后相当于 $C$ 条 $(a_ix_i,b_ix_i^2),0\le x_i\le L$ 的线段做 min-plus 卷积,做 Mincowski 和即可。
\end{frame}

\begin{frame}{QOJ6660}
  给你一棵带权树,根为 $0$,每个点 $i$ 有一个 $A_i$ 和 $B_i$,意味着如果你从 $i$ 出发走距离为 $d$ 的路程,需要 $A_i+d\times B_i$ 的代价,允许换乘任意多次,请计算从根开始出发走到其它点的最小代价。

  $N\le 10^5,0\le A_i\le 10^{12},0\le B_i\le 10^6,1\le W_i\le 10^6$
\end{frame}
\begin{frame}
  首先显然如果我们要换乘,那么 $B_i$ 一定是越来越小的。
  
  所以考虑按照 $B_i$ 排序,从大到小加入。

  但是这个树上问题还是很难刻画距离,所以考虑套一层动态点分治。

  然后对于每一层分治中心,可以维护一个凸包 $f(x)$,表示到分治中心的距离为 $x$ 的情况下,走到这个点的最小代价。

  然后更新形如插入一条线段,用李超树维护即可。由于是全局插入,所以时间复杂度是 $O(n\log^2 n)$。

  也可以用一个单调栈维护凸包然后二分查询。
\end{frame}
\begin{frame}{QOJ3272}
  给定长为 $n$ 的序列 $a$,进行 $q$ 次操作

  1. 给定 $v$,全局 ckmin

  2. 将所有 $a_i\gets a_i+i$

  3. 查询区间和

  $n,q \le 2\times 10^5,0\le a_i,v_i\le 10^{12}$
\end{frame}
\begin{frame}
  直观地想,$a$ 一定是越来越递增。

  对于已经递增的段,我们可以二分找到一个分界点,然后把 ckmin 变成区间覆盖,操作 2 即为区间加等差数列。总之是好做的。
  
  注意到一段只要被操作 1 影响过一次,那么由于后续操作只会使其递增,所以其内部一定是递增的。

  那么我们可以把整个序列划为两部分:$S$ 和 $[1,n]\backslash S$,其中 $S$ 表示被操作 1 影响过至少一次的子序列。
  
  那么我们只需要考虑一个元素何时加入 $S$,考虑二分。

  记 $b_i$ 表示第 $i$ 次 1 操作前 2 操作的数量。

  相当于查询是否 $\exists j\in[1,mid]$,使 $a_i+b_ji\ge v_j$。

  转写为判断 $a_i\ge \min_{j\le mid}\{v_j-b_ji\}$,可以维护一个 $(b_j,v_j)$ 的凸包,每次拿一个斜率为 $i$ 的线去切。

  所以考虑整体二分,然后就做完了。时间复杂度 $O(n\log n)$。由于斜率单调,所以可以双指针维护。
\end{frame}
\begin{frame}
  鸣谢:[广告位招租]
\end{frame}

\end{document}
